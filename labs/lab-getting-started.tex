\section{Lab: Getting started}
\topics{Definitions, functions, basic arithmetic expressions}

The purpose of this lab is twofold: firstly, you will set up the tools that you will be using as part of the module and become acquainted with the basics of how they work; secondly you will encounter, compile, and improve an existing Haskell program.

% The goal of this first practical is twofold: firstly, you will learn to write some very simple Haskell programs and secondly, you will be able to familiarise yourself with the tools we use as part of this module, such as \bashIn{stack} to compile and test programs and \bashIn{git} for version control. The module guide contains instructions on how to use both programs.

\task[ex:setup]{If you have not done so yet, you should set up your Haskell development environment now (see \Cref{sec:department-setup} and \Cref{sec:home-setup}). Most of the tools you will need are pre-installed on the departmental machines, but if you are using one of the lab machines you must open a shell (with the Terminal application on the lab machines) and run the following command to make the \bashIn{stack} tool available on your user account (this only needs to be done once):}
\begin{minted}{bash}
$ /modules/cs141/haskell-setup.sh
\end{minted}

\taskLine

\task{There is some skeleton code for most of the lab sessions, including this one. You can obtain the code for this practical by cloning it from GitHub:}
\begin{minted}{bash}
$ git clone https://github.com/fpclass/lab-getting-started
\end{minted}
By default, this will create a folder named \texttt{\small lab-getting-started} in the current working directory (your home directory, by default) with the skeleton code in it. Once you have cloned the repository, you may wish to verify that \bashIn{stack} compiles it without any problems:
\begin{minted}{text}
$ cd lab-getting-started
$ stack run
\end{minted}
If everything goes well, you should see some output along the lines of:
\begin{minted}{text}
lab-getting-started-1.0.0.0: configure (lib)
Configuring lab-getting-started-1.0.0.0
lab-getting-started-1.0.0.0: build (lib)
Preprocessing library lab-getting-started-1.0.00
[1 of 1] Compiling LabGettingStarted      (src/LabGettin...)
lab-getting-started-1.0.0.0: copy/register
Installing library in ...
Registering lab-getting-started-1.0.0.0
\end{minted}
A window should open with a lovely cat picture. If so, congratulations! The \texttt{\small stack} tool is correctly installed and works as expected -- you are now ready to work on the exercises! If you are not seeing a window with a lovely cat picture, ask one of the lab tutors for assistance.

\taskLine 

There is a \texttt{\small src/LabGettingStarted.hs} file in the \texttt{\small lab-getting-started} directory which contains some definitions responsible for rendering the lovely cat picture. You should open that file in your text editor of choice (see \Cref{ch:tools} for information about the text editors available to you). If you are using Atom, the \bashIn{haskell-setup.sh} script you ran earlier will already have installed some Haskell-related plugins.

\task[ex:open]{Open the \texttt{\small src/LabGettingStarted.hs} file in your preferred text editor. Take a look at the definition of \haskellIn{drawing}. At this point, you are not expected to understand everything you see and the purpose of this exercise is simply to experiment with changing code that is already there. Not everything you try may work, but that's okay! Here are some examples of things you could try before moving on to the next exercise sheet:}

\begin{itemize}
	\item Turn the spinning cat into a spinning dog.
	\item Make the goose moonwalk away from the ducks.
	\item Make the spinning cat spin the other way.
	\item Make the goose run twice as fast.
	\item Hide the dog behind a giant duck.
\end{itemize}

There are a number of pictures available, namely \texttt{cat}, \texttt{dog}, \texttt{duck}, \texttt{goose}, and \texttt{blank}. The standard mathematical operators (\haskellIn{+}, \haskellIn{-}, \haskellIn{*}) are also available, so you can perform calculations on values.

A number of useful operators and expressions related to positioning and transforming images are also available:

\begin{table}[H]
\centering
\begin{tabular}{ll}
Function / Operator       & Description                                          \\ \hline
\texttt{img1 <|> img2}    & Puts \mintinline{haskell}{img2} to the right of \mintinline{haskell}{img1}. \\
\texttt{img1 <-> img2}    & Puts \mintinline{haskell}{img1} above \mintinline{haskell}{img2}.           \\
\texttt{img1 <@> img2}    & Puts \mintinline{haskell}{img1} in front of \mintinline{haskell}{img2}.     \\
\texttt{scale sf img2}    & Scales \mintinline{haskell}{img} by scale factor \mintinline{haskell}{sf}.  \\
\texttt{offset tx ty img} & Moves \mintinline{haskell}{img}, \mintinline{haskell}{tx} pixels right and \mintinline{haskell}{ty} pixels up. \\
\texttt{rotate deg img}   & Rotates \mintinline{haskell}{img} by \mintinline{haskell}{deg} degrees.     \\
\texttt{mirror img}       & Reflects \mintinline{haskell}{img} through the y-axis.
\end{tabular}
\end{table}

Notice that the right hand side of \haskellIn{animation} depends on the parameter to the \haskellIn{animation} function, \haskellIn{t}, which is the number of ticks the animation has been running for.
