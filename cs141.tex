%%%%%%%%%%%%%%%%%%%%%%%%%%%%%%%%%%%%%%%%%%%%%%%%%%%%%%%%%%%%%%%%%%%%%%%%%%%%%%%%
%% LaTeX sources for The Guide to Functional Programming
%% Michael B. Gale (m.gale@warwick.ac.uk)
%%
%% This work is licensed under the Creative Commons
%% Attribution-NonCommercial-ShareAlike 2.0 UK: England & Wales License. To
%% view a copy of this license, visit
%% http://creativecommons.org/licenses/by-nc-sa/2.0/uk/ or send a letter to
%% Creative Commons, PO Box 1866, Mountain View, CA 94042, USA.
%%%%%%%%%%%%%%%%%%%%%%%%%%%%%%%%%%%%%%%%%%%%%%%%%%%%%%%%%%%%%%%%%%%%%%%%%%%%%%%%

\documentclass[12pt,a4paper,twoside,fleqn]{report}

\usepackage{geometry}
\usepackage[latin1]{inputenc}
\usepackage{amsmath}
\usepackage{amsfonts}
\usepackage{amssymb}
\usepackage{graphicx}
\usepackage{fancyeq}
\usepackage{fancyhdr}
\usepackage[explicit]{titlesec}
\usepackage{color}
\usepackage{longtable}
\usepackage{array, booktabs}
\usepackage{colortbl}
\usepackage{wrapfig}
\usepackage{pgfplots}
\usepackage[strict]{changepage}
\usepackage{graphbox}

% Minted -------------------------------

\usepackage{minted}
\usepackage[
backgroundcolor = gray!5
, hidealllines=true
]{mdframed}

\surroundwithmdframed{minted}
\usemintedstyle{lovelace}

\setminted[text]{fontsize=\small}
\setminted[haskell]{fontsize=\small}
\setminted[bash]{fontsize=\small}
\newcommand{\haskellIn}[1]{\mintinline[fontsize=\small]{haskell}{#1}}
\newcommand{\bashIn}[1]{\mintinline[fontsize=\small]{bash}{#1}}

% Tikz -------------------------------

\usepackage{tikz}
\usetikzlibrary{arrows}

\tikzset{
	treenode/.style = {align=center, inner sep=0pt, text centered,
		font=\sffamily},
	arn_n/.style = {treenode, circle, white, font=\sffamily\bfseries, draw=black,
		fill=black, text width=1.5em},
	arn_r/.style = {treenode, circle, red, draw=red,
		text width=1.5em, very thick},
	arn_x/.style = {treenode, circle, draw=white,
		minimum width=1.5em, minimum height=1.5em},
	cls/.style = { treenode, rectangle, red, draw=red,
		text width=1.5em, very thick, minimum width=1.5em, minimum height=1.5em }
}

% UK English -------------------------------
\usepackage[UKenglish]{babel}
\usepackage[UKenglish]{isodate}

% Hyperref -------------------------------
\usepackage{hyperref}
\hypersetup{
    colorlinks=true,
    linkcolor=black,
    urlcolor=black,
    citecolor=black
}

% cleveref
\usepackage[nameinlink]{cleveref}

\crefname{figure}{figure}{figures} % cleveref by default has `fig.'
\Crefname{figure}{Figure}{Figures}

\crefname{section}{section}{sections} % cleveref by default has `fig.'
\Crefname{section}{Section}{Sections}

% Fonts -------------------------------
\usepackage[]{FiraSans}

% Palatino (font)
\usepackage{mathpazo}
\linespread{1.05}         % Palatino needs more leading (space between lines)
\usepackage[T1]{fontenc}

% some format settings
% for hard-bound final submission, use:
%\setlength{\oddsidemargin}{4.6mm}     % 30 mm left margin - 1 in
% for soft-bound version and techreport, use instead:
\setlength{\oddsidemargin}{-0.4mm}    % 25 mm left margin - 1 in
\setlength{\evensidemargin}{-0.4mm}
\setlength{\topmargin}{-5.4mm}        % 20 mm top margin - 1 in
\setlength{\textwidth}{160mm}         % 20/25 mm right margin
\setlength{\textheight}{237mm}        % 20 mm bottom margin
\setlength{\headheight}{5mm}
\setlength{\headsep}{5mm}
\setlength{\parindent}{0mm}
\setlength{\parskip}{\medskipamount}
\renewcommand\baselinestretch{1.2} % thesis format (not needed for techreport)
% don't let large figures hijack entire pages
\renewcommand\topfraction{.9}
\renewcommand\textfraction{.1}
\renewcommand\floatpagefraction{.8}

\usepackage[nomap]{FiraMono}

\usepackage{microtype}
\DisableLigatures[f]{encoding = *, family = tt* }

\author{Michael B. Gale}


\definecolor{gray75}{gray}{0.75}
\newcommand{\hsp}{\hspace{20pt}}

\titleformat{name=\chapter}[hang]{}{}{0cm}{%
	\protect\thispagestyle{fancy}
	\begin{center}
		\large $\lambda$.\thechapter \\
		\huge \textsc{#1}
	\end{center}
	%\chaptermark{\chapter}
}
\titleformat{name=\chapter,numberless}[hang]{}{}{0cm}{%
	\begin{center}
		\huge \textsc{#1}
	\end{center}
	%\chaptermark{\chapter}
}

%\titleformat{\subsection}{format}{label}{0em}{before-code}

\titleformat{\section}
	{\bfseries\large}
	{\llap{\parbox{1.5cm}{\thesection\hfill}}#1}
	{0em}
	{\bfseries}

\titleformat{\subsection}
{\bfseries}{\llap{\parbox{1.5cm}{\thesubsection\hfill}}#1}{0em}{\bfseries}


% misc. commands
\newcounter{TaskCounter}

\newcommand{\question}[1]{\vspace{-0.7cm}{\footnotesize \emph{#1}} \vspace{0.25cm}}
\newcommand{\topics}[1]{\vspace{-0.5cm}{\footnotesize \emph{Topics}: #1} \vspace{0.25cm}}
\newcommand{\task}[2][]{\indent\llap{\parbox{1.5cm}{\refstepcounter{TaskCounter}\label{#1}{\firamedium Ex\theTaskCounter}\hfill}}#2}
\newcommand{\solution}[2]{\indent\llap{\parbox{1.5cm}{\textbf{Ex#1}\hfill}}#2}
\newcommand{\taskLine}{\bigskip\hrule\bigskip}



% titles, dates, etc. of lectures and labs

% lecture 1
\newcommand{\lectureOneTitle}{Introduction}
\newcommand{\lectureOneDate}{7 January}
\newcommand{\lectureOneQuestion}{What is functional programming and why should we learn it?}
\newcommand{\lectureOneTopics}{Overview of programming paradigms \& models of computation, examples of applications of functional programming, module overview, and recommended texts.}

% lecture 2
\newcommand{\lectureTwoTitle}{Definitions \& functions}
\newcommand{\lectureTwoDate}{8 January}
\newcommand{\lectureTwoQuestion}{How do we write simple programs in Haskell?}
\newcommand{\lectureTwoTopics}{Definitions, basic arithmetic expressions, string values, boolean values, functions, using built-in functions, and basic pattern matching.}

% lecture 3
\newcommand{\lectureThreeTitle}{Basic types}
\newcommand{\lectureThreeDate}{9 January}
\newcommand{\lectureThreeQuestion}{How does the compiler prevent us from writing bad software?}
\newcommand{\lectureThreeTopics}{Basic types, function types, parametric polymorphism, lists, and pairs.}

% lecture 3b
\newcommand{\lectureThreeBTitle}{Lists}
\newcommand{\lectureThreeBDate}{14 January}
\newcommand{\lectureThreeBQuestion}{How do we use lists in Haskell?}
\newcommand{\lectureThreeBTopics}{Constructing lists, pattern-matching on lists, and list comprehensions.}

% lecture 4
\newcommand{\lectureFourTitle}{Type classes}
\newcommand{\lectureFourDate}{15 January}
\newcommand{\lectureFourQuestion}{How can we restrict polymorphism and overload functions?}
\newcommand{\lectureFourTopics}{Ad-hoc polymorphism via type classes, built-in type classes, and type class constraints.}

% lecture 5
\newcommand{\lectureFiveTitle}{Recursive functions}
\newcommand{\lectureFiveDate}{16 January}
\newcommand{\lectureFiveQuestion}{How do we express loops without mutable state?}
\newcommand{\lectureFiveTopics}{Writing recursive functions for basic types (numbers, lists/strings), defining built-in functions ourselves.}

% lecture 6
\newcommand{\lectureSixTitle}{Higher-order functions}
\newcommand{\lectureSixDate}{21 January}
\newcommand{\lectureSixQuestion}{Can we write functions which abstract over common behaviours?}
\newcommand{\lectureSixTopics}{Higher-order functions such as \haskellIn{map}, \haskellIn{filter}, etc., recursion primitives such as \haskellIn{foldr} and \haskellIn{foldl}.}

% lecture 7
\newcommand{\lectureSevenTitle}{Data types \& type aliases}
\newcommand{\lectureSevenDate}{22 January}
\newcommand{\lectureSevenQuestion}{How can we define our own types in Haskell?}
\newcommand{\lectureSevenTopics}{Type aliases, data types, data constructors, pattern matching on custom data constructors, recursion on values of custom types.}

% lecture 8
\newcommand{\lectureEightTitle}{Coursework 1 briefing}
\newcommand{\lectureEightDate}{23 January}
\newcommand{\lectureEightQuestion}{What is the first coursework about?}
\newcommand{\lectureEightTopics}{Demonstration of what the completed coursework will do, five-guess algorithm, introduction to the skeleton code.}

% lecture 9
\newcommand{\lectureNineTitle}{Lazy evaluation}
\newcommand{\lectureNineDate}{28 January}
\newcommand{\lectureNineQuestion}{What order are programs evaluated in?}
\newcommand{\lectureNineTopics}{Strict and lazy evaluation, calling conventions, benefits and disadvantages, and examples of lazy evaluation.}

% lecture 9b
\newcommand{\lectureNineBTitle}{\emph{Fun with} testing}
\newcommand{\lectureNineBDate}{29 January}
\newcommand{\lectureNineBQuestion}{What tools are there for testing and how do we use them?}
\newcommand{\lectureNineBTopics}{Unit testing and property-based testing in Haskell.}

% lecture 10
\newcommand{\lectureTenTitle}{Reasoning about programs}
\newcommand{\lectureTenDate}{30 January}
\newcommand{\lectureTenQuestion}{Can we use formal reasoning techniques to prove properties about our programs?}
\newcommand{\lectureTenTopics}{Equational reasoning, proofs by induction.}

% lecture 11
\newcommand{\lectureElevenTitle}{Reasoning about programs (cont.)}
\newcommand{\lectureElevenDate}{4 February}
\newcommand{\lectureElevenQuestion}{Can we use formal reasoning techniques to prove properties about our programs?}
\newcommand{\lectureElevenTopics}{Equational reasoning, proofs by induction.}

% lecture 11b
\newcommand{\lectureElevenBTitle}{\emph{Fun with} constructive induction}
\newcommand{\lectureElevenBDate}{5 February}
\newcommand{\lectureElevenBQuestion}{Can we use formal reasoning techniques to calculate more efficient programs?}
\newcommand{\lectureElevenBTopics}{Constructive induction.}

% lecture 12
\newcommand{\lectureTwelveTitle}{Functors \& applicative functors}
\newcommand{\lectureTwelveDate}{6 February}
\newcommand{\lectureTwelveQuestion}{Are there any useful design patterns in functional programming?}
\newcommand{\lectureTwelveTopics}{Functors, applicative functions, applications of applicative functors.}

% lecture 12b
\newcommand{\lectureTwelveBTitle}{Functors \& applicative functors (cont.)}
\newcommand{\lectureTwelveBDate}{11 February}
\newcommand{\lectureTwelveBQuestion}{Are there any useful design patterns in functional programming?}
\newcommand{\lectureTwelveBTopics}{Functors, applicative functions, applications of applicative functors.}

% lecture 12c
\newcommand{\lectureTwelveCTitle}{\emph{Fun with} applicative functors}
\newcommand{\lectureTwelveCDate}{12 February}
\newcommand{\lectureTwelveCQuestion}{What can we do with applicative functors?}
\newcommand{\lectureTwelveCTopics}{Applicative functors in action.}

% lecture 12d
\newcommand{\lectureTwelveDTitle}{Coursework 2 briefing}
\newcommand{\lectureTwelveDDate}{13 February}
\newcommand{\lectureTwelveDQuestion}{What is the second coursework about?}
\newcommand{\lectureTwelveDTopics}{Demonstration of what the completed coursework will do, semantics of the programming language, introduction to the skeleton code.}

% lecture 13
\newcommand{\lectureThirteenTitle}{Foldables}
\newcommand{\lectureThirteenDate}{18 February}
\newcommand{\lectureThirteenQuestion}{Are there any other useful design patterns in functional programming?}
\newcommand{\lectureThirteenTopics}{\haskellIn{Foldable} type class, its motivation, and examples.}

% lecture 14
\newcommand{\lectureFourteenTitle}{Sequential composition}
\newcommand{\lectureFourteenDate}{19 February}
\newcommand{\lectureFourteenQuestion}{How do structure programs in which one part of a program relies on the result of another part?}
\newcommand{\lectureFourteenTopics}{Some functions for the sequential composition of \texttt{\small Maybe} values.}

% lecture 15
\newcommand{\lectureFifteenTitle}{Sequential composition (cont.)}
\newcommand{\lectureFifteenDate}{25 February}
\newcommand{\lectureFifteenQuestion}{Are there other examples of sequential composition?}
\newcommand{\lectureFifteenTopics}{Some functions for the sequential composition of \texttt{\small State} values and some laws for sequential composition.}

% lecture 16
\newcommand{\lectureSixteenTitle}{\emph{Fun with} sequential composition}
\newcommand{\lectureSixteenDate}{26 February}
\newcommand{\lectureSixteenQuestion}{How is sequential composition used in practice?}
\newcommand{\lectureSixteenTopics}{Sequential composition in action.}

% lecture 17
\newcommand{\lectureSeventeenTitle}{Input and output}
\newcommand{\lectureSeventeenDate}{27 February}
\newcommand{\lectureSeventeenQuestion}{Can we write impure programs in a pure programming language?}
\newcommand{\lectureSeventeenTopics}{The \texttt{\small IO} monad.}

% lecture 18
\newcommand{\lectureEighteenTitle}{Type promotion \& GADTs}
\newcommand{\lectureEighteenDate}{4 March}
\newcommand{\lectureEighteenQuestion}{How can we encode more information in types?}
\newcommand{\lectureEighteenTopics}{Phantom types, GADTs, singleton types, pattern matching with GADTs.}

% lecture 18b
\newcommand{\lectureEighteenBTitle}{\emph{Fun with} IO}
\newcommand{\lectureEighteenBDate}{5 March}
\newcommand{\lectureEighteenBQuestion}{What do Haskell programs that make use of IO look like?}
\newcommand{\lectureEighteenBTopics}{The \texttt{\small IO} monad in action.}

% lecture 19
\newcommand{\lectureNineteenTitle}{Type families}
\newcommand{\lectureNineteenDate}{6 March}
\newcommand{\lectureNineteenQuestion}{How can we perform computation at the type-level?}
\newcommand{\lectureNineteenTopics}{Closed and open type families.}

% lecture 20
\newcommand{\lectureTwentyTitle}{Type-level programming}
\newcommand{\lectureTwentyDate}{11 March}
\newcommand{\lectureTwentyQuestion}{How do we make type-level programming practical in Haskell?}
\newcommand{\lectureTwentyTopics}{Singletons, proxies, and reification.}

% lecture 21
\newcommand{\lectureTwentyOneTitle}{\emph{Fun with} type-level programming}
\newcommand{\lectureTwentyOneDate}{12 March}
\newcommand{\lectureTwentyOneQuestion}{What are some examples of how type-level programming is used?}
\newcommand{\lectureTwentyOneTopics}{Type-level programming in action.}

% lecture 22
\newcommand{\lectureTwentyTwoTitle}{Conclusions}
\newcommand{\lectureTwentyTwoDate}{13 March}
\newcommand{\lectureTwentyTwoQuestion}{What have we learnt about functional programming?}
\newcommand{\lectureTwentyTwoTopics}{Summary of the module, information about the exam, and other general information.}

% practical 1
\newcommand{\practicalOneTitle}{Getting started}
\newcommand{\practicalOneDate}{7-11 January}
\newcommand{\practicalOneAims}{You will learn how to use some of the tools that we will be using as part of this module.}

% practical 2
\newcommand{\practicalTwoTitle}{Types \& list comprehensions}
\newcommand{\practicalTwoDate}{14-18 January}
\newcommand{\practicalTwoAims}{This lab teaches you to understand type errors and how to fix them. You will also learn about list comprehensions.}

% practical 3
\newcommand{\practicalThreeTitle}{Recursive \& higher-order functions}
\newcommand{\practicalThreeDate}{21-25 January}
\newcommand{\practicalThreeAims}{By the end of this lab, you should be able to solve problems by writing recursive and higher-order functions.}

% practical 4
\newcommand{\practicalFourTitle}{User-defined types}
\newcommand{\practicalFourDate}{\parbox{2.2cm}{28 January-\linebreak 1 February}}
\newcommand{\practicalFourAims}{You will define your own types and functions which work with them.}

% practical 5
\newcommand{\practicalFiveTitle}{Lazy evaluation and equational reasoning}
\newcommand{\practicalFiveDate}{4-8 February}
\newcommand{\practicalFiveAims}{The goal of this lab is for you to be able to write programs which make effective use of lazy evaluation, such as for backtracking or infinite data. You will also prove some properties about your programs using equational reasoning and structural induction.}

% practical 6
\newcommand{\practicalSixTitle}{Functors}
\newcommand{\practicalSixDate}{11-15 February}
\newcommand{\practicalSixAims}{You will write programs using functors.}

% practical 6b
\newcommand{\practicalSixBTitle}{Applicative functors}
\newcommand{\practicalSixBDate}{18-22 February}
\newcommand{\practicalSixBAims}{You will write programs using applicative functors.}

% practical 7
\newcommand{\practicalSevenTitle}{Foldables}
\newcommand{\practicalSevenDate}{\parbox{2.4cm}{25 February-\linebreak 1 March}}
\newcommand{\practicalSevenAims}{In this lab, you will write programs using foldables.}

% practical 8
\newcommand{\practicalEightTitle}{Effectful programs}
\newcommand{\practicalEightDate}{4-8 March}
\newcommand{\practicalEightAims}{You will write programs using monads, define your own instances of the \haskellIn{Monad} type class, and reason about monad laws.}

% practical 9
\newcommand{\practicalNineTitle}{Type-level programming}
\newcommand{\practicalNineDate}{11-15 March}
\newcommand{\practicalNineAims}{You should be able to write simple programs at the type-level using GADTs and type families.}

\AtBeginDocument{\addtocontents{toc}{\protect\thispagestyle{fancy}}}

\begin{document}
\fancypagestyle{empty}{\fancyhf{}}
\pagestyle{empty}
\pagenumbering{roman} 

\renewcommand{\headrulewidth}{0pt}
\renewcommand{\footrulewidth}{0pt}

\begin{titlepage}
	\begin{center}
		
{\Huge \textit{The guide to}} \\[0.2cm]
{\Huge \textbf{Functional Programming}} \\[0.2cm]

\vfill

\scalebox{20.0}{$\lambda$}

\vfill 

{\LARGE Michael B. Gale} \\[0.1cm]
{\large \href{mailto:m.gale@warwick.ac.uk}{m.gale@warwick.ac.uk}}

\vspace{2cm}

{2018/19}
\end{center}
\end{titlepage}

\cleardoublepage
%\setcounter{page}{1}

\fancyhf{}
\fancyhead[LE, RO]{\emph{Functional Programming (CS141)}}
\fancyhead[LO, RE]{\emph{Michael B. Gale}}
%\fancyhead[RE,LO]{Guides and tutorials}
%\fancyfoot[CE,CO]{\leftmark}
\fancyfoot[LE,RO]{\thepage}
\pagestyle{fancy}
\thispagestyle{fancy}
\newgeometry{
	tmargin=2.5cm,
	textwidth=155mm, 
	textheight=247mm,
	headheight=5mm,
	headsep=5mm,
	inner=30mm
}

\tableofcontents


\cleardoublepage
\pagenumbering{arabic}

%%%%%%%%%%%%%%%%%%%%%%%%%%%%%%%%%%%%%%%%%%%%%%%%%%%%%%%%%%%%%%%%%%%%%%%%%%%%%%%%
%% LaTeX sources for The Guide to Functional Programming
%% Michael B. Gale (m.gale@warwick.ac.uk)
%%
%% This work is licensed under the Creative Commons
%% Attribution-NonCommercial-ShareAlike 2.0 UK: England & Wales License. To
%% view a copy of this license, visit
%% http://creativecommons.org/licenses/by-nc-sa/2.0/uk/ or send a letter to
%% Creative Commons, PO Box 1866, Mountain View, CA 94042, USA.
%%%%%%%%%%%%%%%%%%%%%%%%%%%%%%%%%%%%%%%%%%%%%%%%%%%%%%%%%%%%%%%%%%%%%%%%%%%%%%%%

\chapter{Overview}

\emph{Functional Programming} is an optional module which follows on from modules such as CS118, CS132, or equivalents in other departments where you have learnt to write programs in the imperative style in languages such as C and Java. However, C and Java are just two of many programming languages and object-oriented programming is just one of many programming paradigms. You may think of programming languages as tools: a hammer is different from a screwdriver and both serve different purposes which they are good at. Programming languages are the same: different languages exist for different purposes and it is easier or harder to accomplish certain tasks in one or the other. To be a good programmer, you need to know which tools are at your disposal and when to use them.

In this module, you will learn about the functional programming paradigm. No prior programming knowledge is required and this module is suitable for most scientists. We will use Haskell, which is a lazy, purely functional programming language. Writing programs in Haskell is very different than writing programs in languages like Java and over the course of this module you will learn how to do that. In turn, this adds a powerful tool to your programming arsenal, you will gain a much deeper understanding of programming, and skills from this module can be applied in other languages, functional or not. In other words, you will become a better programmer!

This document serves as a companion to the module by giving you an overview of all the major components, including guidance on how to use the different tools you will encounter as part of this module. You can also find the coursework specifications as well as exercises for all of the labs in this guide.
\input{books.tex}
\section{Timeline}
\label{sec:timeline}

This module is comprised of approximately 30 lectures, 10 labs, 2 pieces of coursework, and an exam. This section contains a chronological schedule of all of these components. Note that the schedule may be subject to changes due to \emph{e.g.} staff illness or other unforeseen circumstances. Each lecture aims to answer a specific question, which is shown in the timeline. You can test your understanding by asking yourself that question after each lecture and checking that you can answer it. 

There are typically three lectures per week. The definite timetable is available from the timetabling website\footnote{\url{https://timetablingmanagement.warwick.ac.uk/sws1920/}}, on the module website, or you can view your personal timetable on Tabula as well. 

\newcommand{\foo}{\makebox[0pt]{\textbullet}\hskip-0.5pt\vrule width 1pt\hspace{\labelsep}}

\newcommand{\LectureEntry}[4]{#1 & \begin{tabular}{p{11cm}}
		\textbf{#2} \\
		\emph{#3} \\
		#4
\end{tabular}}
\newcommand{\LabEntry}[3]{#1 & \begin{tabular}{p{11cm}}
		\textbf{#2} \\
		#3
\end{tabular}}

\begingroup
%\begin{table}
\newcommand{\oldarraystrech}{\arraystretch}
	\renewcommand\arraystretch{1.4}\vskip-1.5ex
	\begin{longtable}{@{\,}r <{\hskip 2pt} !{\foo} >{\raggedright\arraybackslash}p{12cm}}
		\addlinespace[1.5ex]
		\LabEntry{6-10 January}{Week 1 exercises}{Relevant exercises for this week are \emph{Getting started}, \emph{Basic types}, \emph{Functions}, \emph{Pattern matching}.} \\
		\LectureEntry{6 January}{Lecture 1: Introduction}{What is functional programming and why should we learn it?}{Overview of programming paradigms \& models of computation, examples of applications of functional programming, module overview, and recommended texts.} \\
		\LectureEntry{7 January}{Lecture 2: Definitions \& functions}{How do we write simple programs in Haskell?}{Definitions, basic arithmetic expressions, string values, boolean values, functions, using built-in functions, and basic pattern matching.} \\
		\LectureEntry{8 January}{Lecture 3: Basic types}{How does the compiler prevent us from writing bad software?}{Basic types, function types, parametric polymorphism, lists, and pairs.} \\
		
		\LabEntry{13-17 January}{Week 2 exercises}{Relevant exercises for this week are \emph{Using the standard library}, \emph{Lists}, \emph{List comprehensions}, \emph{Recursive functions}, \emph{Higher-order functions}.} \\
		\LectureEntry{13 January}{Lecture 4: Lists}{How do we use lists in Haskell?}{Constructing lists, pattern-matching on lists, and list comprehensions.} \\
		\LectureEntry{14 January}{Lecture 5: Recursive functions}{How do we express loops without mutable state?}{Writing recursive functions for basic types (numbers, lists/strings), defining built-in functions ourselves.} \\
		\LectureEntry{15 January}{Lecture 6: Higher-order functions}{Can we write functions which abstract over common behaviours?}{Higher-order functions such as \haskellIn{map}, \haskellIn{filter}, etc., recursion primitives such as \haskellIn{foldr} and \haskellIn{foldl}.} \\
		
		\LabEntry{20-24 January}{Week 3 exercises}{Relevant exercises for this week are \emph{Data types}, \emph{Type aliases}, and \emph{Using general libraries}.} \\
		\LectureEntry{20 January}{Lecture 7: Data types \& type aliases}{How can we define our own types in Haskell?}{Type aliases, data types, data constructors, pattern matching on custom data constructors, recursion on values of custom types.} \\
		\LectureEntry{21 January}{Lecture 8: \emph{Fun with} functions}{What are some more examples of functions?}{Extended examples of using and defining functions.} \\
		\LectureEntry{22 January}{Lecture 9: Coursework I briefing}{What is the first coursework about?}{Demonstration of what the completed coursework will do, explanation of the rules, introduction to the skeleton code.} \\
		
		\LabEntry{27-31 January}{Week 4 exercises}{Relevant exercises for this week are \emph{Type classes}.} \\
		\LectureEntry{27 January}{Lecture 10: Type classes}{How can we restrict polymorphism and overload functions?}{Ad-hoc polymorphism via type classes, built-in type classes, and type class constraints.} \\
		\LectureEntry{28 January}{Lecture 11: \emph{Fun with} type classes}{How are type classes used in Haskell?}{Examples of type class and their instances.} \\
		\LectureEntry{29 January}{Lecture 12: Testing}{What tools are there for testing and how do we use them?}{Unit testing, property-based testing, and code coverage in Haskell.} \\
		
		\LabEntry{3-7 February}{Week 5 exercises}{Recommended exercises for this week are \emph{Proofs}.} \\
		\LectureEntry{3 February}{Lecture 13: Reasoning about programs}{Can we use formal reasoning techniques to prove properties about our programs?}{Equational reasoning, proofs by induction.} \\
		\LectureEntry{4 February}{Lecture 14: Reasoning about programs (cont.)}{Can we use formal reasoning techniques to prove properties about our programs?}{Equational reasoning, proofs by induction.} \\
		\LectureEntry{5 February}{Lecture 15: Constructive induction}{Can we use formal reasoning techniques to calculate more efficient programs?}{Using induction to calculate faster functions.} \\
		
		\hline
		6 February & \begin{tabular}{p{13cm}}
			\textbf{Deadline: Coursework I} 
		\end{tabular}\\
		\hline
		
		\LabEntry{10-14 February}{Week 6 exercises}{Relevant exercises for this week are \emph{Lazy evaluation}, \emph{Foldables}, and \emph{Functors}.} \\
		\LectureEntry{10 February}{Lecture 16: Lazy evaluation}{What order are programs evaluated in?}{Strict and lazy evaluation, calling conventions, benefits and disadvantages, and examples of lazy evaluation.} \\
		\LectureEntry{11 February}{Lecture 17: Foldables}{Are there any useful design patterns in functional programming?}{\haskellIn{Foldable} type class, its motivation, and examples.} \\
		\LectureEntry{12 February}{Lecture 18: Functors \& applicative functors}{Are there any other useful design patterns in functional programming?}{Functors, applicative functions, applications of applicative functors.} \\
		
		\LabEntry{17-21 February}{Week 7 exercises}{Relevant exercises for this week are \emph{Applicatives}.} \\
		\LectureEntry{17 February}{Lecture 19: Functors \& applicative functors (cont.)}{Are there any other useful design patterns in functional programming?}{Functors, applicative functions, applications of applicative functors.} \\
		\LectureEntry{18 February}{Lecture 20: \emph{Fun with} applicative functors}{What can we do with applicative functors?}{Applicative functors in action.} \\
		\LectureEntry{19 February}{Lecture 21: Coursework II briefing}{What is the second coursework about?}{Demonstration of what the completed coursework will do, semantics of the programming language, introduction to the skeleton code.} \\
		
		\LabEntry{24-28 February}{Week 8 exercises}{Recommended exercises for this week are \emph{Effectful programming}.} \\
		\LectureEntry{24 February}{Lecture 22: Sequential composition}{How do structure programs in which one part of a program relies on the result of another part?}{Some functions for the sequential composition of \texttt{\small Maybe} values.} \\
		\LectureEntry{25 February}{Lecture 23: Sequential composition (cont.)}{Are there other examples of sequential composition?}{Some functions for the sequential composition of \texttt{\small State} values and some laws for sequential composition} \\
		\LectureEntry{26 February}{Lecture 24: \emph{Fun with} sequential composition}{How is sequential composition used in practice?}{Sequential composition in action.} \\
		
		\LabEntry{2-6 March}{Week 9 exercises}{Relevant exercises for this week are \emph{Input \& output} and \emph{Kinds}.} \\
		\LectureEntry{2 March}{Lecture 25: Input and output}{Can we write impure programs in a pure programming language?}{The \texttt{\small IO} monad.} \\
		\LectureEntry{3 March}{Lecture 26: \emph{Fun with} IO}{What do Haskell programs that make use of IO look like?}{The \texttt{\small IO} monad in action.} \\
		\LectureEntry{4 March}{Lecture 27: Type promotion \& GADTs}{How can we encode more information in types?}{Phantom types, GADTs, singleton types, pattern matching with GADTs.} \\
		
		\LabEntry{9-13 March}{Week 10 exercises}{Relevant exercises for this week are \emph{GADTs} and \emph{Type families}.} \\
		\LectureEntry{9 March}{Lecture 28: Type families}{How can we perform computation at the type-level?}{Closed and open type families.} \\
		\LectureEntry{10 March}{Lecture 29: \emph{Fun with} type-level programming}{What are some examples of how type-level programming is used?}{Type-level programming in action.} \\
		\LectureEntry{11 March}{Lecture 30: Conclusions}{What have we learnt about functional programming?}{Summary of the module, information about the exam, and other general information.} \\
		\hline
		13 March & \begin{tabular}{p{13cm}}
			\textbf{Deadline: Coursework II} 
		\end{tabular}\\
		\hline
		Term 3 & \begin{tabular}{p{13cm}}
			\textbf{Revision lectures}  \\
			Student-selected topics from the previous lectures.
		\end{tabular}\\
		Term 3 & \begin{tabular}{p{13cm}}
			\textbf{Exam}  \\
			2 hours. Answer any four out of six questions.
		\end{tabular}
	\end{longtable}
%\end{table}
\endgroup

\pagebreak
\section{Coursework (15\% + 25\%)}

There are two pieces of coursework which you will have to complete during Term 2 and submit through Tabula. You will receive feedback for the first coursework before the second coursework is due.

\subsection{Mastermind (15\%)}

\paragraph{Description} You have to implement Knuth's five-guess algorithm to complete a program in which a human player takes turns playing the board game mastermind against the computer. 

\paragraph{Aims} This coursework is designed to test your ability to write basic Haskell programs using built-in functions, translate pseudocode into a functional programming language, write recursive functions, and make effective use of lazy evaluation. 

\subsection{Scratch clone (25\%)}

\paragraph{Description} Scratch is a popular tool for teaching programming to children. For this coursework, you have to implement an interpreter for a simple programming language which is used to complete a clone of Scratch. 

\paragraph{Aims} This tests your ability to make use of type classes, data types, and functional design patterns such as monads.
\section{Exam (60\%)}

The exam is worth 60\% of the module and takes place in Term 3. You will have to answer any four out of six questions. Each question is worth 25 marks. Past exam papers as well as a sample exam paper are available on the module website. You may also be able to find exam papers from before 2017/18, but note that their format and content are significantly different.

In general, I prefer to set exam questions which require you to use your understanding of functional programming and Haskell to solve problems of varying difficulties. There may be few or no questions that are bookwork and there will be no lengthy essay questions for you to answer. To lessen the need for memorisation, you are permitted to take this guide into the exam with you. A reference of the Haskell standard library may be found in \Cref{ch:prelude}.

There will be three revision lectures in Term 3, the dates of which are to be confirmed. You are welcome to suggest topics for these.
\input{guidance.tex}
\input{lecture-notes.tex}

\cleardoublepage
\chapter{Labs}

This chapter contains exercises for the CS141 lab sessions. Each section in this chapter contains exercises which roughly correspond to the lecture content for the respective week or the one preceding it. Due to the timetabling of labs this year, you may have your timetabled lab session before the respective content is covered in the lectures. Do not worry about this though and just work on the exercises that you know how to do. Alternatively you can also work on the coursework. CS141 labs are very informal and should primarily be fun! There is no expectation that you complete all exercises every week and you are welcome to work on anything you like at whatever pace you like. We do take a register to keep track of attendance, but do not worry if you cannot make it to a lab because of illness or some other reason. If you have missed a lab, you are welcome to come to another lab session if you would like provided that there are spaces.

%\input{lectures/lecture1.tex}
%\input{lectures/lecture2.tex}
\input{labs/lab1.tex} \newpage

%\input{lectures/lecture3.tex}
%\input{lectures/lecture4.tex}
\section{Lab 2: \practicalTwoTitle}
\topics{Basic types, function types, parametric polymorphism, lists, pairs, ad-hoc polymorphism via type classes, built-in type classes, type class constraints, and list comprehensions.}

This practical is about \emph{types} in Haskell. You can obtain the code for this lab by cloning the respective repository from GitHub:
\begin{minted}{bash}
$ git clone https://github.com/fpclass/lab2
\end{minted}
In a nutshell, the type of an expression tells you what sort of value an expression eventually evaluates to. Not all expressions can be typed! If an expression cannot be reduced to a normal form, such as \haskellIn{not 7}, then it does not have a type. We use the following notation to say that an expression \texttt{\small expression} has type \texttt{\small type} -- this is referred to as a \emph{typing}:
\begin{center}
	\begin{tabular}{lcl}
		\texttt{\small expression} & \texttt{\small ::} & \texttt{\small type}
	\end{tabular}
\end{center}
Some examples of typings are:
\begin{minted}{haskell}
True                       :: Bool
'a'                        :: Char 
\x -> x                    :: a -> a
\x -> False                :: a -> Bool 
\x -> \y -> y              :: a -> b -> b
if 5 > 6 then 'a' else 'b' :: Char 
42                         :: Num a => a
4 + 8 * 15 - 16            :: Num a => a
(\x -> x) True             :: Bool
\end{minted}
As you can see, the size or complexity of a term does not necessarily correspond to that of a type. It only matters what an expression evaluates to. Some expressions have multiple permissible types. For example, the expression \haskellIn{42} could have type \haskellIn{Int} (the type of signed integers where the precision depends on your platform), \haskellIn{Integer} (the type of arbitrary precision integers), \haskellIn{Num a => a} (a polymorphic type representing all numeric types, \emph{i.e.} instances of the \haskellIn{Num} type class), as well as others.

\subsubsection{Types}

If an expression can be typed, the Haskell compiler can \emph{infer} the \emph{most general} type for us. For example, for numbers such as \haskellIn{42}, the \haskellIn{Int} type is permissible, but \haskellIn{Num a => a} is a more general type. ``More general'' generally means ``more polymorphic''. There is a command in the REPL which we can use to ask for the type of an expression:
\begin{center}
\begin{tabular}{|l|l|}
	\hline 
	\texttt{\small :t EXPRESSION}   & Shows the type of \texttt{\small EXPRESSION}. \\ 
	\hline 
\end{tabular}  
\end{center}

\taskLine

\task{Launch the REPL and try it for yourself by asking for the types of the following expressions:}
\begin{itemize}
	\item \haskellIn{'a'}
	\item \haskellIn{[True, True, False]}
	\item \haskellIn{[1,2,3,4,5]}
	\item \haskellIn{[]}
	\item \haskellIn{\x -> x}
	\item \haskellIn{1+1}
\end{itemize}

\taskLine

\task{Some expressions cannot be reduced to values and therefore do not have a type. Try asking for the types of the following expressions. Each expression will result in the type error which is explained below:} 
\begin{itemize}
\item \haskellIn{not 7} \\
As mentioned in the lecture on type classes, the literal \haskellIn{7} has the most general type \haskellIn{Num a => a}. The Haskell compiler also knows that the \haskellIn{not} function has type \haskellIn{Bool -> Bool} so it expects its argument to be of type \haskellIn{Bool}. Because \haskellIn{Bool} is less polymorphic than \haskellIn{a}, the Haskell compiler deduces that \haskellIn{7} should be of type \haskellIn{Bool}. However, this results in a constraint of \haskellIn{Num Bool} which cannot be resolved because there is no instance of \haskellIn{Num} for the \haskellIn{Bool} type.
\item \haskellIn{[1,True,3]} \\
This results in the same error.
\item \haskellIn{['a', False]} \\
Lists in Haskell are \emph{homogeneous}. That is, they can only contain elements which have the same type. Neither \haskellIn{'a'} nor \haskellIn{False} have a polymorphic type and both have different types. Therefore, the Haskell compiler will tell you that the two types do not match.
\end{itemize}

\taskLine

\task{Recall from the lecture on type classes that the REPL essentially wraps each expression you try to evaluate into a call to the \haskellIn{show} function. Try to evaluate some expressions which \emph{are} well typed, but whose types do not have a \haskellIn{Show} instance. This is usually the case for functions. For example: }
\begin{itemize}
\item \haskellIn{not} \\
Even though \haskellIn{not} is well typed, you get a type error if you try to evaluate it in the REPL. That is because the REPL does not know how to display a result that is a function. The error you get will tell you that there is no \haskellIn{Show} instance for \haskellIn{Bool -> Bool}, the type of \haskellIn{not}. 
\end{itemize}

\taskLine

\subsubsection{List comprehensions}

Haskell has some syntactic sugar for generating lists which is referred to as \emph{list comprehensions}. These expressions are very similar to set comprehensions in mathematics. For this part of the lab, you will learn about their syntax and implement some lists using them.

In the first lecture, you saw the \haskellIn{[1..4]} notation to generate the list of numbers from 1 to 4: \haskellIn{[1,2,3,4]}. In general, the \haskellIn{[n..m]} syntax can be used to denote ranges between any two values \haskellIn{n} and \haskellIn{m}. Note that for this to work, \haskellIn{n} and \haskellIn{m} must have the same type and that type must have an instance of the \haskellIn{Enum} type class. Specifically, \haskellIn{[n..m]} is syntactic sugar for \haskellIn{enumFromTo n m}, a member of the \haskellIn{Enum} type class.

\taskLine

\task{Complete the definitions of \haskellIn{zeroToTen} and \haskellIn{fourToEight} in \texttt{\small Lab2.hs} to implement the lists representing the numbers from 0 to 10 and from 4 to 8 respectively.}

\taskLine

\task{As mentioned above, the \haskellIn{[n..m]} syntax works for any type that is an instance of \haskellIn{Enum}. Complete the definition of \haskellIn{lowercase} in \texttt{\small Lab2.hs} to implement the list of lower-case characters from \haskellIn{'a'} to \haskellIn{'z'}.}

\taskLine

List comprehensions can also be used to generate lists from other lists, just like set comprehensions are used to generate sets from other sets. For example, the following expression is a list comprehension which generates the list of numbers that are double the numbers from 0 to 10:
\begin{minted}{haskell}
[2*n | n <- [0..10]]
\end{minted}
The \haskellIn{n <- [0..10]} part in this example is referred to as a \emph{generator}. Given some list on the right of \haskellIn{<-}, it loops through all the elements of that list and binds them to \haskellIn{n} one after the other. The part on the left of the \haskellIn{|} is what is used to generate an element of the resulting list, for all values obtained from the right of the \haskellIn{|}. So, for this example, the resulting list is \haskellIn{[0,2,4,6,8,10,12,14,16,18,20]}.

\taskLine

\task{Using a list comprehension, complete the definition of \haskellIn{powersOfTwo}, which calculates the powers of two for the factors from 1 to 10. The exponentiation operator in Haskell is \haskellIn{^}.} 

\taskLine

\task{Using a list comprehension, complete the definition of \haskellIn{factorials}, which calculates the list of factorials for the numbers 1 to 10.}

\taskLine

List comprehensions may contain more than one generator. If there is more than one generator, all subsequent generators loop through their elements for every element in the preceding generator. You can think of these as nested loops:
\begin{minted}{haskell}
[x*y | x <- [1..3], y <- [1..4]]
\end{minted}
In this example, there are three elements in the list used by the first generator. Each element in that list is multiplied with every element in the list used by the second generator. Therefore, the result is a list of 12 elements where the first four elements are multiples of one, the next four are multiples of two, and the last four are multiples of three: \haskellIn{[1,2,3,4,2,4,6,8,3,6,9,12]}.

\taskLine

\task{Using a list comprehension with two generators, define \haskellIn{coords} to be a list of coordinates where the top left corner of the coordinate system is \haskellIn{(0,0)} and the bottom right is \haskellIn{(10,10)}.}

\taskLine

\task{Generators after the first in a list comprehension may also refer to variables that are bound by preceding generators. Use this to complete the definition of \linebreak \haskellIn{noMoreThanFive} which should be a list of all pairs \haskellIn{(x,y)} such that \haskellIn{x} is a number from 0 to 5 and \haskellIn{y} is also a number from 0 to 5, unless $x+y > 5$: \texttt{\small [(0,0), (0,1), (0,2), (0,3), (0,4), (0,5), (1,0), (1,1), (1,2), (1,3), (1,4), \linebreak (2,0), ...]}}.

\taskLine

Finally, a list comprehension may contain predicates to determine which elements produced by generators should be used for elements in the resulting list. For example, the following list comprehension generates all numbers which are factors of some number \haskellIn{n}. The \haskellIn{mod} function from the standard library calculates the remainder of two numbers:
\begin{minted}{haskell}
factors :: Int -> [Int]
factors n = [x | x <- [1..n], mod n x == 0]
\end{minted} 

\taskLine

\task{Complete the definition of \haskellIn{evens} which should be the list of all numbers from 0 to 100 which are even.}

\taskLine

\task{Complete the definition of \haskellIn{multiples} which, given some natural number \haskellIn{n}, should produce the list of all factors of 3 and 5 in the range from 0 to \haskellIn{n} (inclusive). For example, \haskellIn{multiples 10} should evaluate to \haskellIn{[0,3,5,6,9,10]}.}

\taskLine

\task{Finally, ensure that all your definitions are correct by running all of the unit tests with \bashIn{stack test} in a terminal.}

\taskLine \newpage

%\input{lectures/lecture5.tex}
%\input{lectures/lecture6.tex}
\section{Lab 3: \practicalThreeTitle}
\topics{Recursive \& higher-order functions and monoids.}

This practical is mainly about recursive and higher-order functions in Haskell. You can obtain the skeleton code for this lab by cloning the repository from GitHub:
\begin{minted}{bash}
$ git clone https://github.com/fpclass/lab3
\end{minted}
In a nutshell, recursive functions are functions which are defined in terms of themselves. Higher-order functions are functions which take other functions as arguments or return functions.

\taskLine

\task[task:higher-order-typings]{For each of the following statements, discuss with someone (friend, tutor, rubber duck, etc.) whether it is true or false:
\begin{enumerate}
\item A function of type \mintinline{text}{a -> b -> c} returns a function.
\item A function of type \mintinline{text}{(a -> b) -> Int} returns a function. 
\item A function of type \mintinline{text}{(Int, Bool) -> Char} is higher-order.
\item A function of type \mintinline{text}{a -> a} can be a higher-order function.
\end{enumerate}
}

\taskLine

\subsubsection{Recursive and higher-order functions}

In the fifth lecture, you learnt how to write functions using \emph{explicit recursion}, as demonstrated in the definition of \haskellIn{and} below:
\begin{minted}{haskell}
and :: [Bool] -> Bool
and []     = True 
and (x:xs) = x && and xs
\end{minted}
You also learnt how to use higher-order functions to abstract over common patterns in the sixth lecture. For example, you could define \haskellIn{and} with the help of \haskellIn{foldr} as the following, in which case the recursion is not explicit:
\begin{minted}{haskell}
and :: [Bool] -> Bool 
and = foldr (&&) True
\end{minted} 

\taskLine

\task[task:elem-explicit]{Using explicit recursion, complete the definition of 
	
\haskellIn{elem :: Eq a => a -> [a] -> Bool}

which should determine whether some value of type \texttt{\small a} is contained in a list of values of type \texttt{\small a}. For example, \haskellIn{elem 4 [4,8,15,4]} should evaluate to \haskellIn{True} and \haskellIn{elem 7 [4,8,15,4]} should evaluate to \haskellIn{False}.}

\task[task:elem-composition]{Now try to define \haskellIn{elem} entirely in terms of the following standard library functions: \haskellIn{not}, \haskellIn{null}, \haskellIn{filter} with an appropriate predicate, and function composition.} 

\taskLine

\task{Using explicit recursion, complete the definition of}

\haskellIn{maximum :: Ord a => [a] -> a}

which should find the greatest element of the list given as argument. For example, \haskellIn{maximum [1,2,3,2,1]} should evaluate to \haskellIn{3}. You may assume that \haskellIn{maximum} will never be called with the empty list so you do not need to define an equation for that case.

\task{The \haskellIn{foldr1 :: (a -> a -> a) -> [a] -> a} function behaves like \haskellIn{foldr}, except that it assumes lists always have at least one element. It therefore only requires a binary operation and a list as arguments. Implement \haskellIn{maximum} using \haskellIn{foldr1}.}

\taskLine 

\task{Complete the definition of

\haskellIn{intersperse :: a -> [a] -> [a]}

which should separate elements of a list with some separator of the same type as the elements of the list. For example, \haskellIn{intersperse '|' "WITTER"} should evaluate to \haskellIn{"W|I|T|T|E|R"}.

\emph{Hint}: you may find it useful to define an additional function to help you.
}

\taskLine

\task{Complete the definition of \haskellIn{any}, which should test whether at least one element of the list given as argument satisfies the predicate. For example, \haskellIn{any even [1,2,3]} should evaluate to \haskellIn{True}.}

\task{Can you define \haskellIn{elem} using only \haskellIn{any} with an appropriate predicate?}

\taskLine

\task{Complete the definition of \haskellIn{all}, which should test whether all elements of the list given as argument satisfy the predicate. For example, \haskellIn{all odd [1,3,5]} should evaluate to \haskellIn{True}.}

\taskLine

\task{Implement the \haskellIn{flip} function which, given a function with two parameters as argument, produces a function with the parameters flipped. Note: there are no unit tests for this function, but your implementation will be correct if it is well typed according to the typing in the skeleton code.}

\taskLine 

\task{Complete the definition of 
	
\haskellIn{takeWhile :: (a -> Bool) -> [a] -> [a]} 

which generalises the \haskellIn{take} function to a predicate: \haskellIn{takeWhile} should take elements from the list argument and return them while the predicate holds. For example, \haskellIn{takeWhile (< 3) [1,2,3,2,1]} should evaluate to \haskellIn{[1,2]}. 
}

\taskLine 

\task{Complete the definition of \haskellIn{zipWith}, which generalises \haskellIn{zip}. While \haskellIn{zip} puts elements from two lists into pairs in a resulting list, \haskellIn{zipWith} requires a function of type \haskellIn{a -> b -> c} as argument which combines elements of type \texttt{\small a} from the first list and elements of type \texttt{\small b} from the second list into elements of type \texttt{\small c} for the resulting list. For example, \haskellIn{zipWith replicate [1,2,3] ['a', 'b', 'c']} should evaluate to \haskellIn{[['a'], ['b', 'b'], ['c', 'c', 'c']]}.}

\taskLine 

\task{Complete the definition of \haskellIn{groupBy}, which should group elements of a list according to some predicate. For example, \haskellIn{groupBy (==) [1,2,2,3,4,4,1]} should evaluate to \haskellIn{[[1], [2,2], [3], [4,4], [1]]}.}

\taskLine

\task{Complete the definition of
	
\haskellIn{subsequences :: [a] -> [[a]]}
	
which should find all possible subsequences of the argument. For example, evaluating \haskellIn{subsequences "abc"} should result in a list such as \linebreak \haskellIn{["", "a", "b", "c", "ab", "bc", "ac", "abc"]}. The order of the elements in the resulting list does not matter.
} 

\taskLine

\task{Complete the definition of
	
\haskellIn{permutations :: Eq a => [a] -> [[a]]}
	
which should find all possible permutations of the argument. For example, evaluating \haskellIn{permutations "abc"} should result in a list such as \linebreak \haskellIn{["abc", "acb", "bac", "bca", "cab", "cba"]}. The order of the elements in the resulting list does not matter.
} 

\task{Could you implement \haskellIn{permutations} without the \haskellIn{Eq} constraint on \texttt{\small a}?}

\taskLine 

\subsubsection{Monoids}

A \emph{monoid} is an algebraic structure which consists of a unit value and an associative, binary operation. In Haskell, we can declare a type class for types which are monoids:
\begin{minted}{haskell}
class Monoid a where
    mempty  :: a
    mappend :: a -> a -> a
    mconcat :: [a] -> a
\end{minted}
Here, \texttt{mempty} represents the unit value and \texttt{mappend} represents the associative, binary operation. Instances of the \texttt{Monoid} type class should obey the following \emph{monoid laws}:
\begin{displaymath}
\begin{array}{lcrcl}
\textbf{Left identity} &\qquad & \mathit{mappend}~\mathit{mempty}~x & = & x \\
\textbf{Right identity} &\qquad & \mathit{mappend}~x~\mathit{mempty} & = & x \\
\textbf{Associativity} & \qquad & \mathit{mappend}~x~(\mathit{mappend}~y~z) & = & \mathit{mappend}~(\mathit{mappend}~x~y)~z \\
\textbf{Concatenation} & \qquad & \mathit{mconcat} & = & \mathit{foldr}~\mathit{mappend}~\mathit{mempty}
\end{array}
\end{displaymath}
We say that a type \emph{forms} a monoid if there is an instance of the \haskellIn{Monoid} type class for it which obeys the monoid laws. 

The \haskellIn{mconcat} function shown above is not necessary for a type to be a monoid, but it generalises the ordinary \haskellIn{concat} function on lists of lists and can easily be implemented with the help of \haskellIn{mappend} and \haskellIn{mempty}.

\taskLine 

\task{The \haskellIn{mconcat} function does nothing specific to any particular type. Specify a default implementation for the \haskellIn{mconcat} function in the declaration of the \haskellIn{Monoid} type class so that it obeys the fourth monoid law.}

\task{Does it matter whether you use \haskellIn{foldr} or \haskellIn{foldl} for the implementation of \haskellIn{mconcat}?}

\taskLine

\task{Implement \haskellIn{mempty} and \haskellIn{mappend} of the \haskellIn{Monoid} instance for \haskellIn{Int} so that they obey the monoid laws. Note that there are two possible implementations which satisfy the monoid laws -- can you think of both?}

\taskLine 

\task{Implement \haskellIn{mempty} and \haskellIn{mappend} of the \haskellIn{Monoid} instance for \texttt{\small [a]} so that they obey the monoid laws.}

\taskLine 

\task{Functions of type \texttt{\small a -> b} form a monoid if \texttt{\small b} is a monoid. Implement \haskellIn{mempty} and \haskellIn{mappend} of the \haskellIn{Monoid} instance for \texttt{\small a -> b} so that they obey the monoid laws. Note that there are no unit tests for this task as it would require tests for function equality.}

\taskLine 


%\input{lectures/lecture7.tex}
%\input{lectures/lecture8.tex}
\input{labs/lab4.tex}

%\input{lectures/lecture9.tex}
%\input{lectures/lecture10.tex}
\section{Lab 5: \practicalFiveTitle}
\topics{Lazy evaluation, infinite data structures, equational reasoning, constructive induction.}

This practical is about lazy evaluation, infinite data structures, equational reasoning, and constructive induction. You can obtain the skeleton code for the first part of this lab by cloning the respective repository from GitHub:
\begin{minted}{bash}
$ git clone https://github.com/cs256/lab5
\end{minted}
Do not worry if you do not manage to finish all the exercises in an hour, there are more than even I could do in that time! However, the exercises are useful practice in the long run.

\subsubsection{Infinite data structures}

In the lecture on lazy evaluation, you saw that Haskell supports infinite data structures such as infinite lists. This is possible because, at runtime, variables in Haskell are just pointers to closures. For example, we saw the following definition in the lecture:
\begin{minted}{haskell}
from :: Int -> [Int]
from n = n : from (n+1)
\end{minted}
Recall that the Haskell compiler transforms expressions which appear as arguments to functions into let-bound definitions:
\begin{minted}{haskell}
from :: Int -> [Int]
from n = let ns = from (n+1) in n : ns
\end{minted}
Thus, when \texttt{\small from n} is called for some \texttt{\small n}, a new closure for \texttt{\small ns} is allocated on the heap. Because of lazy evaluation, the call to \haskellIn{from (n+1)} is not evaluated immediately. It is only evaluated when the tail of \texttt{\small n~na:~ns} is needed and then the closure represented by \texttt{\small ns} is updated with the result of \haskellIn{from (n+1)}. The call to \haskellIn{from (n+1)} will allocate yet another closure for \haskellIn{from (n+2)} which is only evaluated when the tail of the tail of \haskellIn{from n} is needed.

\taskLine 

\task[task:ones]{Complete the definition of \haskellIn{ones} which should represent an infinite list where all elements are \haskellIn{1}.}

\taskLine

You also saw that the following definition for the infinite list of Fibonacci numbers can be implemented elegantly and efficiently as the following in Haskell:
\begin{minted}{haskell}
fibs :: [Integer]
fibs = 1 : 1 : zipWith (+) fibs (tail fibs)
\end{minted}
The infinite list represented by \haskellIn{fibs} can be generated in linear time for the number of elements requested. This is possible because \haskellIn{fibs} is transformed into the following:
\begin{minted}{haskell}
fibs :: [Integer]
fibs = let xs = tail fibs 
           ys = zipWith (+) fibs xs
           zs = 1 : ys
       in 1 : zs
\end{minted}
All closures involved in this definition can be updated with their results. Therefore, \haskellIn{fibs} refers to a cons cell (a kind of closure) where the head is \haskellIn{1} and the closure pointed to by \texttt{\small zs} is the tail. The closure represented by \texttt{\small zs} is also a cons cell where the head is \haskellIn{1} and the tail is pointed to by \texttt{\small ys}. The closure represented by \texttt{\small ys} will be updated with the result of \haskellIn{zipWith (+) fibs xs} when more than two elements are requested from \haskellIn{fibs}, \emph{i.e.} the tail of \texttt{\small zs} is inspected by a \haskellIn{case}-expression somewhere. The \haskellIn{zipWith} function, in turn, allocates more closures when called, thus recursively adding more cons cells as needed.

However, not all closures can be updated. Suppose we want to generalise the definition of \haskellIn{fibs} to sequences with arbitrary seed values \texttt{\small x} and \texttt{\small y}:
\begin{minted}{haskell}
foos :: Integer -> Integer -> [Integer]
foos x y = x : y : zipWith (+) (foos x y) (tail (foos x y))
\end{minted}
This will be exponentially slow, because Haskell will not update the closure for \haskellIn{foos} with the list that is being generated. As a result, every call to \haskellIn{foos x y} will result in the list being generated from the beginning. This makes sense, because \haskellIn{foos} is parametrised over \texttt{\small x} and \texttt{\small y} and it would be wrong to update the closure for \haskellIn{foos} with the list generated for some specific \texttt{\small x} and \texttt{\small y}. 

\taskLine

\task[task:arbitrary-sequence]{Implement \haskellIn{foos'} to produce the same sequence as \haskellIn{foos}, but in linear time. \emph{Hint}: you need to find a way to allocate a closure which is specific to some \texttt{\small x} and \texttt{\small y} and can be updated.}

\task[task:lazy-bench]{You can run \bashIn{stack bench} to compare the time complexities of \haskellIn{foos} and \haskellIn{foos'}. If you have done everything right, you should see a bar graph with similar growth as the graph shown in \Cref{fig:foos-complexity} in the \texttt{lab5.html} file which is generated by \bashIn{stack bench}.}

\pgfplotsset{width=7cm,compat=1.15}
\begin{fancyfig}{Time complexities of \haskellIn{foos} and \haskellIn{foos'}}{fig:foos-complexity}
\begin{center}	
	\begin{tikzpicture} \begin{axis}
	[
	    legend style={at={(1.5,1)},
	    	anchor=north,legend columns=-1},
		enlargelimits=0.15,
		ylabel=Time ($\mu$s),
		xlabel=Elements
	]
	\addplot+ [
	sharp plot,
	] coordinates {(10,0.994) (20,4.48)
		(30,10.2)};
	\addplot+ [
	sharp plot,
	] coordinates {(10,0.300) (20,0.610) 
		(30,0.919)};
	\legend{\texttt{foos},\texttt{foos'}}
	\end{axis}
	\end{tikzpicture}
\end{center}
\end{fancyfig}

\taskLine

\subsubsection{Induction on natural numbers}

Recall that we have already proved the following (monoidal) properties about natural numbers in the lecture on equational reasoning:
\begin{displaymath}
\begin{array}{lcrcl}
	\textbf{Left unit} &\qquad & \mathit{add}~Z~x & = & x \\
	\textbf{Right unit} &\qquad & \mathit{add}~x~Z & = & x \\
	\textbf{Associativity} & \qquad & \mathit{add}~x~(\mathit{add}~y~z) & = & \mathit{add}~(\mathit{add}~x~y)~z 
\end{array}
\end{displaymath}

\taskLine

\task[task:succ-comm]{Prove the following property about addition just by rewriting one side of the equation until you end up with the other side:
\begin{displaymath}
\forall n :: \mathit{Nat} . S~n = \mathit{add}~(S~Z)~n
\end{displaymath}
You should show each step of the proof along with a comment to say what you have done at that particular step, as shown in the lecture and the proofs in \Cref{sec:lecture-10}.
}

\task[task:successor-commutes]{Prove the following property about addition by induction on $n$. For this proof, you will need to make use of some of the other properties you know about $\mathit{add}$, including the one you proved for Exercise \ref{task:succ-comm}.
\begin{displaymath}
\forall n~m :: \mathit{Nat} . \mathit{add}~(S~n)~m = \mathit{add}~n~(S~m)
\end{displaymath}
}

\task[task:add-commutes]{Finally, using all the properties you know about $\mathit{add}$ so far, prove that $\mathit{add}$ commutes:
\begin{displaymath}
\forall n~m :: \mathit{Nat} . \mathit{add}~n~m = \mathit{add}~m~n
\end{displaymath}
}

\taskLine

\subsubsection{Induction on lists}

In Haskell, the \haskellIn{reverse} function can be defined as follows:
\begin{minted}{haskell}
reverse :: [a] -> [a]
reverse []     = []
reverse (x:xs) = reverse xs ++ [x]
\end{minted}
This makes use of the \haskellIn{(++)} operator, which is defined as follows:
\begin{minted}{haskell}
(++) :: [a] -> [a] -> [a]
[]     ++ ys = ys 
(x:xs) ++ ys = x : (xs ++ ys)
\end{minted}

\taskLine

\task[task:reverse-identity-left]{Prove that $(\append)$ has a left identity by rewriting the following equation:
\begin{displaymath}
\forall \mathit{xs} :: \hslist{a}~. \quad \hslist{} \append \mathit{xs} = \mathit{xs}
\end{displaymath}}

\task[task:reverse-identity-right]{Prove that $(\append)$ has a right identity by induction on $\mathit{xs}$:
\begin{displaymath}
\forall \mathit{xs} :: \hslist{a}~. \quad \mathit{xs} \append \hslist{} = \mathit{xs}
\end{displaymath}}

\task[task:append-assoc]{Prove that $(\append)$ is associative by induction on $\mathit{xs}$:
	\begin{displaymath}
	\forall \mathit{xs}~\mathit{ys}~\mathit{zs} :: \hslist{a}~. \quad \mathit{xs} \append (\mathit{ys} \append \mathit{zs}) = (\mathit{xs} \append \mathit{ys}) \append \mathit{zs}
	\end{displaymath}}

\task[task:reverse-preserves]{Prove that $\mathit{reverse}$ preserves singleton lists by rewriting the following equation:
\begin{displaymath}
\forall \mathit{x} :: a~. \quad \mathit{reverse}~\hslist{x} = \hslist{x}
\end{displaymath}}

\task[task:reverse-distributes-over-append]{Prove that $\mathit{reverse}$ distributes over $(\append)$ by induction on $\mathit{xs}$. You will need some of the properties you have proved so far about $(\append)$ and $\mathit{reverse}$.
\begin{displaymath}
\forall \mathit{xs}~\mathit{ys} :: \hslist{a}~. \quad \mathit{reverse}~(\mathit{xs} \append \mathit{ys}) = \mathit{reverse}~\mathit{ys} \append \mathit{reverse}~\mathit{xs}
\end{displaymath}}

\task[task:reverse-of-reverse]{Prove the following property about $\mathit{reverse}$ by induction on $\mathit{xs}$. You will need some of the properties you have proved so far about $(\append)$ and $\mathit{reverse}$.
\begin{displaymath}
\forall \mathit{xs} :: \hslist{a}~. \quad \mathit{reverse}~(\mathit{reverse}~\mathit{xs}) = \mathit{xs}
\end{displaymath}}

\taskLine

\subsubsection{Constructive induction}

It is possible to use induction to \emph{calculate} faster function definitions. This is referred to as \emph{constructive induction}. For example, consider our current definition of \haskellIn{reverse}:
\begin{minted}{haskell}
reverse :: [a] -> [a]
reverse []     = []
reverse (x:xs) = reverse xs ++ [x]
\end{minted}
This definition is inefficient because the \haskellIn{(++)} operator runs in $\mathcal{O}(n)$ time where $n$ is the length of the first argument. The \haskellIn{reverse} function runs in quadratic time as a result. We can do better by combining the behaviour of \haskellIn{reverse} and \haskellIn{(++)} into one new function which does both. We begin by expressing this idea as the following specification:
\begin{minted}{haskell}
rev :: [a] -> [a] -> [a]
rev xs ys = reverse xs ++ ys
\end{minted}
The \haskellIn{rev} function takes two lists as arguments, reverses the first and then appends the second list to it. Our goal is now to use induction to come up with a new definition for \haskellIn{rev} which neither uses \haskellIn{reverse} nor \haskellIn{(++)}. We do this by taking the current definition and performing induction on \texttt{\small xs}. Recall that there are two cases for induction on lists: the empty list (a base case) and cons (a recursive case). We can write down a skeleton for the new definition of \haskellIn{rev} by covering these two cases:
\begin{minted}{haskell}
rev :: [a] -> [a] -> [a]
rev []     ys = ???
rev (x:xs) ys = ???
\end{minted} 

\task[task:rev-empty]{Replace the \texttt{\small ???} in the first equation by reducing \haskellIn{reverse [] ++ ys} as much as possible. The resulting expression should neither contain \haskellIn{reverse} nor \haskellIn{(++)}. \emph{Hint}: you only need to apply \haskellIn{reverse} and \haskellIn{(++)} until you are left with an expression which cannot be reduced any further. This expression can then be used as the right-hand side of the first equation above.}

\task[task:rev-cons]{Replace the \texttt{\small ???} in the second equation by reducing \haskellIn{reverse (x:xs) ++ ys} as much as possible. The resulting expression should neither contain \haskellIn{reverse} nor \haskellIn{(++)}. \emph{Hint}: in this case, you can use the specification \texttt{\small rev xs ys = reverse xs ++ ys} as induction hypothesis.}

\taskLine

%\pagebreak

%\begin{center}
%	\vspace*{4cm}
%	\textbf{This page is intentionally left blank.}
%	\vfill
%\end{center} \newpage

%\input{lectures/lecture11.tex}
%\input{lectures/lecture12.tex}
\section{Lab 6: \practicalSixTitle}
\topics{Functors.}

This practical is about functors. As usual, you can obtain the skeleton code for this this lab by cloning the respective repository from GitHub:
\begin{minted}{bash}
$ git clone https://github.com/fpclass/lab6
\end{minted}
 \newpage

%\input{lectures/lecture13.tex}
%\input{lectures/lecture14.tex}
\input{labs/lab7.tex} \newpage

%\input{lectures/lecture15.tex}
%\input{lectures/lecture16.tex}
\section{Lab 8: \practicalSevenTitle}
\topics{Foldables.}

This practical is about foldables. As usual, you can obtain the skeleton code for this this lab by cloning the respective repository from GitHub:
\begin{minted}{bash}
$ git clone https://github.com/fpclass/lab8
\end{minted} \newpage

%\input{lectures/lecture17.tex}
%\input{lectures/lecture18.tex}
\input{labs/lab9.tex} \newpage

\section{Lab 10: \practicalNineTitle}
\topics{Kinds, phantom types, GADTs, singleton types, pattern matching with GADTs, data type promotion, closed and open type families.}

This final lab is all about type-level programming. The skeleton code can be obtained as usual with the following command:
\begin{minted}{bash}
$ git clone https://github.com/fpclass/lab10
\end{minted}
There are some commands that are supported by the REPL which you may find useful for type-level programming:
\begin{center}
	\begin{tabular}{|l|l|}
		\hline 
		\texttt{\small :k TYPE}   & Infers the kind of \texttt{\small TYPE}. \\ 
		\hline 
		\texttt{\small :kind!~TYPE}  & Reduces \texttt{\small TYPE} to a normal form. \\ 
		\hline 
	\end{tabular} 
\end{center}

\taskLine 

\task{GHC provides type-level booleans out-of-the-box for us. With \texttt{\small -XDataKinds} enabled, the \texttt{\small Bool} type is automatically promoted to a kind and its data constructors, \haskellIn{True} and \haskellIn{False}, are automatically promoted to types. Try running the following commands in the REPL:}
\begin{itemize}
	\item \texttt{\small :k Bool} 
	\item \texttt{\small :t True} 
	\item \texttt{\small :k True} 
\end{itemize}

\task{Define a closed type family \texttt{\small Not} which performs boolean negation at the type-level.}

\task{Once defined, you should be able to experiment with the REPL commands that are described above. Try the following:}
\begin{itemize}
	\item \texttt{\small :k Not}
	\item \texttt{\small :k Not True}
	\item \texttt{\small :kind!~Not True}
\end{itemize}

\taskLine

\task{Modify the definition of \texttt{\small SBool} to define a singleton type for booleans. This type should have kind \texttt{\small Bool -> *} and two constructors named \haskellIn{STrue} and \haskellIn{SFalse} with appropriate types. Once you have defined this type, verify in the REPL that the type has the correct kind and that the constructors have the correct types. The unit tests will also ensure that \haskellIn{STrue} and \haskellIn{SFalse} have the correct types.}

\task{Having a singleton type for booleans is useful as we can now keep track of the value of a boolean variable at the type-level and therefore at compile-time. This allows us to define functions of types such as:}
\begin{minted}{haskell}
inot :: SBool b -> SBool (Not b)
\end{minted}
That is, given a value of type \texttt{\small SBool b} where \texttt{\small b} is a type of kind \texttt{\small Bool} corresponding to the value, \haskellIn{inot} should return a boolean whose value is the negation of \texttt{\small b}. Implement this function now so that we get the expected behaviour:
\begin{minted}{haskell}
inot STrue  ==> SFalse 
inot SFalse ==> STrue
\end{minted}
While this is not very interesting on the term-level, what happens if you ask the REPL for the types of these expressions?
\begin{itemize}
	\item \texttt{\small :t inot STrue}
	\item \texttt{\small :t inot SFalse}
\end{itemize}

\task{Could you define similar functions for other boolean operations?}

\task{Given a type that is known at compile-time, we may wish to convert it to a corresponding value on the value-level. This process is in general known as \emph{reification} and can be accomplished with the help of a suitable type class. We now want to do this for type-level booleans. You are already given the definition of a suitable type class, named \haskellIn{KnownBool}. Implement suitable instances of this type class so that \haskellIn{boolVal} can be used in the following ways:}
\begin{minted}{haskell}
boolVal (Proxy :: Proxy True)  ==> True
boolVal (Proxy :: Proxy False) ==> False
\end{minted}

\taskLine

\task{GHC also provides type-level lists out-of-the-box for us when \texttt{\small -XDataKinds} is enabled. Try the following in the REPL:}
\begin{itemize}
	\item \texttt{\small :k []}
	\item \texttt{\small :t (:)}
	\item \texttt{\small :t []}
	\item \texttt{\small :k '[]}
	\item \texttt{\small :k (:)}
	\item \texttt{\small :k Int :~'[]}
\end{itemize}

\taskLine 

The aim of this last part of the lab is to implement \emph{heterogeneous lists} in Haskell. The ``ordinary'' lists that we have come across in Haskell are homogeneous: that is, every element has the same type. For example, the following is a valid list in Haskell because all elements have the same type:
\begin{minted}{haskell}
[4,8,15,16,23,42] :: [Int]
\end{minted}  
However, the following is not a valid list in Haskell because its elements have different types:
\begin{minted}{haskell}
[True,"Duck"] -- not well typed
\end{minted} 
This is because lists in Haskell need to be parametrised by the element type: the list type constructor \texttt{\small []} has kind \haskellIn{* -> *}. The definition of lists assumes that every element has that type:
\begin{minted}{haskell}
[]  :: [a]
(:) :: a -> [a] -> [a]
\end{minted} 
In order to implement lists where the elements can have different types, we need to be able to parametrise a list by the types of all of its elements. In other words, we need a type constructor of kind \texttt{\small [*] -> *}. That is, a type constructor which requires a list of types as argument.

\task{With the help of type-level lists, complete the definition of \texttt{\small HList} so that it has two constructors: \haskellIn{HNil} which represents an empty, heterogeneous list and \haskellIn{HCons} which adds an element to a heterogeneous list. Some examples of what should work successfully in the REPL once you are done:}
\begin{minted}{haskell}
*Lab10 Lab10> :t HNil
HNil :: HList '[]
*Lab10 Lab10> :t HCons True HNil
HCons True HNil :: HList '[Bool]
*Lab10 Lab10> :t HCons 4 (HCons True HNil)
HCons 4 (HCons True HNil) :: Num a => HList '[a, Bool]
\end{minted}

\task{Implement the \haskellIn{hhead} function, which should work just like \haskellIn{head} does on ordinary lists, but for heterogeneous lists.}

\task{Define suitable instances of the \haskellIn{Show} type class so that we can use the \haskellIn{show} function on heterogeneous lists. For example:}
\begin{minted}{haskell}
show HNil                          ==> "[]"
show (HCons 4 HNil)                ==> "4 : []"
show (HCons "cake" (HCons 4 HNil)) ==> "\"cake\" : 4 : []"
\end{minted}


%\input{lectures/lecture19.tex}

\cleardoublepage

% submission details
\newcommand{\deadlineOneTime}{noon}
\newcommand{\deadlineOneDate}{7 February 2019}
\newcommand{\submissionOneURL}{https://tabula.warwick.ac.uk/coursework/submission/1905b143-7b68-4f61-bf2d-5288934c6253}

%\renewcommand{\instructions}{Due at \emph{\deadlineTime} on \emph{\deadlineDate}.}


\cleardoublepage
\chapter{Coursework I: Mastermind}

The aim of this coursework is to implement the board game \emph{Mastermind} in Haskell with the help of some skeleton code. The game is played by exactly two players: a \emph{codemaker} and a \emph{codebreaker}. At the start of the game, the codemaker makes up a code consisting of four coloured pegs. Pegs are also referred to as symbols. For example:
\begin{center}
    Yellow, Green, Green, Blue
\end{center}
Each colour (symbol) may be used any number of times in the code, as long as the code has no more than four pegs. There are six colours to choose from. The code is \emph{not} disclosed to the codebreaker, whose objective it is to figure out what the code is. The codebreaker does this by repeatedly \emph{guessing} what the code might be. For example, to start the codebreaker might guess the following code at random:
\begin{center}
    Green, Red, Blue, Blue
\end{center}
The codemaker then scores the guess according to the following rules:
\begin{itemize}
    \item For each peg that is in the correct position and has the right colour, the codebreaker scores one coloured marker.
    \item For each peg that is the right colour but in an incorrect position, the codebreaker scores one white marker.
\end{itemize}
For example, for the above guess, the codebreaker would score one white marker for the green peg that is in the wrong position and one coloured marker for the blue peg that is in the right position. The codebreaker does \emph{not} score a white marker for the second blue peg. In other words, at most one point is awarded for each peg in the code. The codebreaker then has to use this score to come up with a new guess for the code, which is then scored again, and so on. Once the codebreaker scores four coloured markers, the game is over and the two players switch roles.

%-----------------------------------------------------------

\section{Getting started}

In order to get started with the coursework, you need to get hold of the skeleton code and ensure that it compiles successfully. 

\subsection{Obtaining the skeleton code}

There are three different ways in which you can obtain the skeleton code for this coursework, which are all explained below alongside their advantages and disadvantages:

\paragraph{Option A: Private fork} By following the GitHub Classroom link below, you can create a private fork of our git repository with the skeleton code. This requires a GitHub account, but has the advantage that you have your own private copy of our repository on GitHub that you can write to. That would allow then you to work easily share your work between machines in the labs and at home:
\begin{center}
	\url{https://classroom.github.com/a/gD7o5fXq}
\end{center}
Once you have accepted the assignment, you can then clone your fork of the skeleton code to your machine with the usual \bashIn{git clone} command where \texttt{\small [username]} is your GitHub username:
\begin{minted}{bash}
$ git clone https://github.com/fpclass/1819-cswk1-[username]
\end{minted}

\paragraph{Option B: Clone} If you do not wish to create a GitHub account or host a copy of your repository there, then you could instead just clone our repository with:
\begin{minted}{bash}
$ git clone https://github.com/fpclass/cswk1
\end{minted}
You will be able to \bashIn{git commit} changes to your local copy of the repository, but you will not be able to \bashIn{git push} them. This is sufficient if you are only planning to work on the coursework from one place (\emph{e.g.} only the lab machines but not your personal computer).

\paragraph{Option C: Archive} If GitHub should be unavailable or you do not have \bashIn{git} installed your machine, you can download a \texttt{\small .zip} file with the skeleton code from the module website.

\subsection{Working with the skeleton code}

You may wish to verify that the code compiles and that all tests fail by entering the \texttt{\small cswk1} directory that was created and running \bashIn{stack test}:
\begin{minted}{bash}
$ cd cswk1
$ stack test
\end{minted}
Running \bashIn{stack test} will compile your code, run a bunch of unit tests on it, and give you a rough indication of how complete your solution is (the more tests pass, the more complete it is). Running \bashIn{stack bench} will run a set of benchmarks on your code. You can also use \bashIn{stack build} to just compile your code and then \bashIn{stack exec mastermind} to run the program. Alternatively, you can run \bashIn{stack repl} to load up the REPL, which is useful for debugging.

The skeleton code contains a bunch of files, most of which you do not need to touch. The most important file is \texttt{\small src/Game.hs} which contains the definitions you will need to complete in order to implement the game. There are some definitions to get you started. Firstly, the number of pegs per code is defined as:
\begin{minted}{haskell}
pegs :: Int 
pegs = 4
\end{minted}
Ideally, your solution should still work even if this number is modified. We represent colours using characters from \texttt{a} to \texttt{f} and refer to them as symbols:
\begin{minted}{haskell}
type Symbol = Char 

symbols :: [Symbol]
symbols = ['a'..'f']
\end{minted}
Again, your solution should continue to work even if you modify how many symbols there are and which characters are used to represent them. A code is a list of symbols:
\begin{minted}{haskell}
type Code = [Symbol]
\end{minted}
Codes are scored using coloured and white markers. We define scores to be pairs of integers where the first component of the pair represents the number of coloured markers and the second component represents the number of white markers:
\begin{minted}{haskell}
type Score = (Int, Int)
\end{minted}
A player is either human or a computer:
\begin{minted}{haskell}
data Player = Human | Computer
\end{minted}
The initial codemaker is defined as a constant:
\begin{minted}{haskell}
codemaker :: Player
codemaker = Human
\end{minted}
You can change this value to determine who goes first. Finally, the computer's first guess is defined as:
\begin{minted}{haskell}
firstGuess :: Code 
firstGuess = "aabb"
\end{minted}
You can change this value to change the computer's first guess, but note that values other than \haskellIn{"aabb"} may cause the computer to take more guesses to crack the code.

%-----------------------------------------------------------

\section{Five-guess algorithm}

Donald Knuth described an algorithm for Mastermind which, for every code with four pegs, takes a computer no more than five guesses to solve. The algorithm works as follows:

\begin{enumerate}
    \item Let $S$ be a set of all possible codes (\haskellIn{"aaaa"}, \haskellIn{"aaab"}, $\ldots$, \haskellIn{"ffff"}).
    \item Let the first guess be \haskellIn{"aabb"}.
    \item Get the codemaker to score your guess.
    \item If the score has four coloured markers, then the guess was correct.
    \item Otherwise, remove all codes from $S$ which would result in a different score. In other words, we know that the code is somewhere in $S$, so it can only be one which results in the same score for the guess as the one we got from the codemaker. 
    \item Find the next guess as follows. If there is only one code left in $S$, use it. Otherwise, for every possible code $c$ (not just those left in $S$):
    \begin{enumerate}
        \item For each possible score $s$ ($(0,1)$, $(0,2)$, $\ldots$, $(4,0)$):
        \begin{enumerate}
            \item Determine how many other codes would be eliminated from $S$. That is, if the next guess were $c$ and it would get a score of $s$, how many codes would that eliminate from $S$ -- i.e. how many codes with different scores would there be?
        \end{enumerate}
    \end{enumerate}
    Choose the code which is guaranteed to eliminate the most options from $S$. This is calculated in the above step by calculating the minimum of eliminations for each code across all the possible scores it might get.
    In the case of multiple codes producing the same number of guaranteed eliminations, a code which is still a member of $S$ should be picked over one which is not.
    \item Go to Step 3.
\end{enumerate}

%-----------------------------------------------------------

\section{Task}

Complete all definitions in \texttt{\small src/Game.hs} so that the game works as described above and that the computer never takes more than five guesses to figure out a code. The following function stubs in \texttt{\small src/Game.hs}  need to be implemented:

\begin{enumerate}
	\item \haskellIn{correctGuess :: Score -> Bool}\\
	This function should determine whether a \haskellIn{Score} value represents a winning guess -- \emph{i.e.} one where the number of coloured markers matches \haskellIn{pegs} and there are no white markers.
	\item \haskellIn{validateCode :: Code -> Bool}\\
	This function should determine whether a given \haskellIn{Code} value is valid: the code should contain \haskellIn{pegs}-many symbols and all the symbols should be elements of \haskellIn{symbols}.
	\item \haskellIn{codes :: [Code]}\\
	This list should contain all possible codes of length \haskellIn{pegs} using elements from \haskellIn{symbols}. There should be no duplicates.
	\item \haskellIn{results :: [Score]}\\
	This list should contains all possible scores for codes of length \haskellIn{pegs}. There should be no duplicates.
	\item \haskellIn{score :: Code -> Code -> Score}\\
	This function should score a code according to the rules described above. This function should be commutative, so that it does not matter whether the code or the guess is given as first argument and vice-versa.
	\item \haskellIn{nextGuess :: [Code] -> Code} \\
	This function should determine the next guess, given the current $S$ represented as a list of codes.
	\item \haskellIn{eliminate :: Score -> Code -> [Code] -> [Code]}\\
	This function should eliminate all codes from a given $S$, represented as a list of codes, with the help of the most recent guess (the \haskellIn{Code} argument) and the score which was obtained for it from the codemaker (the \haskellIn{Score} argument).
\end{enumerate}

%-----------------------------------------------------------

\section{Marking \& submission}

This coursework is worth 15\% of the overall module mark. It will be marked out of 100\% as follows:
\begin{itemize}
\item 20\% for \emph{correctness}. You gain full marks here if all parts of the coursework have been attempted and are correct. You may use \bashIn{stack test} as a rough indication for whether this is the case, but there are some things the unit tests do not test for, so you should play the game and ensure that everything works as described.
\item 20\% for \emph{documented understanding}. You should document your code with comments and explain how it works. You gain full marks if all code is documented and explained sufficiently well so that someone who is unfamiliar with your code can understand it.
\item 20\% for \emph{elegance}. Definitions should be concise and readable, new functions should be introduced where needed, existing library functions used when applicable, etc. 
\item 20\% for \emph{performance and efficiency}. You will do well here if you use sensible data structures and your functions perform as little redundant computation as possible. You can test performance by running \bashIn{stack bench} on different versions of your code to see how they compare. 
\item 20\% for \emph{improvements and extensions}. This is an opportunity for you to demonstrate creativity and advanced understanding. You could achieve this in many different ways, such as adding additional unit tests, functionality, improved algorithms, etc. You may wish to modify \texttt{\small exe/Main.hs} as well as other source files or even add new ones. You could also prove some properties about your game on paper. The amount of marks awarded will depend on the complexity and creativity of your extension(s) and improvement(s).
\end{itemize}
Submit a \texttt{\small .zip} or \texttt{\small .tar.gz} archive of the whole, completed project (not just \texttt{\small Game.hs}) through Tabula by \deadlineOneTime\ on \deadlineOneDate:
\begin{center} 
	\url{\submissionOneURL}
\end{center}


\cleardoublepage

% submission details
\newcommand{\deadlineTwoTime}{noon}
\newcommand{\deadlineTwoDate}{14 March 2019}
\newcommand{\submissionTwoURL}{https://tabula.warwick.ac.uk/coursework/submission/df8197f0-faf5-4d4d-8918-c3a3ab90fb0e}


%\renewcommand{\instructions}{Due at \emph{\deadlineTime} on \emph{\deadlineDate}.}

\cleardoublepage
\chapter{Coursework II: Scratch clone}

Scratch\footnote{\url{https://scratch.mit.edu/}} is a visual programming language designed to teach programming to children in a fun and graphical way. Programs in Scratch are built by arranging blocks that correspond to different syntactic constructs and connecting them like puzzle pieces. The tool is free to use so you can give it a go if you want! To give you an idea of what it looks like, here is a screenshot of Pac-Man built in Scratch running on a Raspberry Pi:

\begin{center}
\includegraphics[width=390px]{cswk/scratch_rpi.png}
\end{center}

The goal of this coursework is to implement a simple clone of Scratch. Our clone will consist of two components:
\begin{enumerate}
    \item A web-based interface which allows users to construct simple programs visually. This is written in JavaScript and is already implemented for you.
    \item A Haskell program which handles the evaluation of such programs. This is partially implemented and you will have to finish it.
\end{enumerate}
Your task is to complete the part of the Haskell program responsible for evaluating programs -- in other words, you have to write an \emph{interpreter}.

An interpreter is a program which, given some representation of a program as argument, evaluates it. To illustrate this idea, a Haskell implementation of an interpreter for a simple expression language is shown below:
\begin{displaymath}
\begin{array}{l}
\begin{array}{lcl}
\mathbf{data}~\mathit{Expr} & = & \mathit{Val}~\mathit{Int} \mid \mathit{Add}~\mathit{Expr}~\mathit{Expr}
\end{array} \\\\
\begin{array}{lcl}
\mathit{eval} & :: & \mathit{Expr} \to \mathit{Int} \\
\mathit{eval}~(\mathit{Val}~n) & = & n\\
\mathit{eval}~(\mathit{Add}~l~r) & = & \mathit{eval}~l + \mathit{eval}~r
\end{array}
\end{array}
\end{displaymath}
Expressions in the language represented by $\mathit{Expr}$ consist of the addition operator and integer values. The $\mathit{eval}$ function is the interpreter for this language, which determines the value of a given expression.

\section{Getting started}

In order to get started with the coursework, you need to get hold of the skeleton code and ensure that it compiles successfully. 

\subsection{Obtaining the skeleton code}

There are three different ways in which you can obtain the skeleton code for this coursework, which are all explained below alongside their advantages and disadvantages:

\paragraph{Option A: Private fork} By following the GitHub Classroom link below, you can create a private fork of our git repository with the skeleton code. This requires a GitHub account, but has the advantage that you have your own private copy of our repository on GitHub that you can write to. That would allow then you to work easily share your work between machines in the labs and at home:
\begin{center}
	\url{https://classroom.github.com/a/5TvwO1Xl}
\end{center}
Once you have accepted the assignment, you can then clone your fork of the skeleton code to your machine with the usual \bashIn{git clone} command where \texttt{\small [username]} is your GitHub username:
\begin{minted}{bash}
$ git clone https://github.com/fpclass/1819-cswk2-[username]
\end{minted}

\paragraph{Option B: Clone} If you do not wish to create a GitHub account or host a copy of your repository there, then you could instead just clone our repository with:
\begin{minted}{bash}
$ git clone https://github.com/fpclass/cswk2
\end{minted}
You will be able to \bashIn{git commit} changes to your local copy of the repository, but you will not be able to \bashIn{git push} them. This is sufficient if you are only planning to work on the coursework from one place (\emph{e.g.} only the lab machines but not your personal computer).

\paragraph{Option C: Archive} If GitHub should be unavailable or you do not have \bashIn{git} installed your machine, you can download a \texttt{\small .zip} file with the skeleton code from the module website.

\subsection{Working with the skeleton code}

The code should compile out of the box. You can test this by running:
\begin{minted}{bash}
$ stack build
\end{minted}
To start the program, you should run the following:
\begin{minted}{bash}
$ stack exec scratch-clone
Starting web server...
Started. Press any key to quit.
\end{minted}
In order to view the user interface, open your web browser and navigate to\footnote{If, for whatever reason, port 8000 is unavailable on your machine, you can change this by modifying the definition of \haskellIn{main} in \texttt{\small src/Main.hs}.}:
\begin{center}
\url{http://localhost:8000/}
\end{center}
You can drag together programs using building blocks from the toolbox on the left. However, if you click ``Evaluate'' at the top right corner of the screen, you will get an error since the interpreter is not yet implemented.

The skeleton code contains a bunch of files, most of which you do not need to touch initially. The most important file is \texttt{\small src/Interpreter.hs} which contains the definitions you will need to complete to get the interpreter to work. There are some definitions to get you started. A program's initial memory is represented as a list of pairs. Each pair represents one variable, consisting of a name of type \haskellIn{String} and a value of type \haskellIn{Int}:
\begin{minted}{haskell}
type Memory = [(String, Int)]
\end{minted}
It is possible for things to go wrong when interpreting a program. There are two sorts of errors which may occur. These are represented by the following data type:
\begin{minted}{haskell}
data Err = DivByZeroError | UninitialisedMemory String
\end{minted}
The types representing the language itself are defined in \texttt{\small src/Language.hs}. You should have a look at this file yourself, but an overview of the most important types is below. A program is a list of statements:
\begin{minted}{haskell}
type Program = [Stmt]
\end{minted}
There are three different forms of statements: 
\begin{minted}{haskell}
data Stmt = AssignStmt String Expr
          | IfStmt Expr [Stmt] [(Expr,[Stmt])] [Stmt]
          | RepeatStmt Expr [Stmt]
\end{minted}
Assignments, represented by the \haskellIn{AssignStmt} constructor, consists of the name of the variable that we are assigning a value to and the expression whose value we should assign to the variable. 

If statements, represented by the \haskellIn{IfStmt}, are more complicated. The first expression is the condition of the ``if'' clause. The list of statements which follows is the code that should be run if the condition is true. The list of pairs of expressions and lists of statements represent ``if else'' clauses. Finally, the last list of statements represents the ``else'' clause.

Repeat statements, represented by the \haskellIn{RepeatStmt} constructor, consist of an expression which determines how many times the repeat loop should be executed and a list of statements which represent the body of the repeat statement.

There are also three forms of expressions:
\begin{minted}{haskell}
data Expr = ValE Int
          | VarE String
          | BinOpE Op Expr Expr
\end{minted}
The \haskellIn{ValE} constructor represents integer values, the \haskellIn{VarE} constructor represents variables, and the \haskellIn{BinOpE} constructor generalises binary operators. The \haskellIn{Op} data type in \texttt{\small src/Language.hs} enumerates all available operators.

\section{Task}

Complete the definition of the \haskellIn{interpret} function in \texttt{\small src/Interpreter.hs} so that all values of type \haskellIn{Program} can be evaluated correctly according to the rules described below. Programs are sequences of statements and should be evaluated in the order in which they are given. We illustrate all rules for the language with screenshots of the GUI and the expected results:

\begin{center}
	\begin{longtable}[t]{|c|p{5cm}|}
		\hline 
		\includegraphics[align=t,width=250px]{cswk/0-empty.png} & 
		If the program is empty as shown in the screenshot, the initial contents of the memory should be returned. \\ \hline 
		\includegraphics[align=t,width=250px]{cswk/1-assignment.png} & 
		Assignment statements should update the memory to the value of their expression. \\ \hline 
		\includegraphics[align=t,width=250px]{cswk/2-loading.png} &
		If a variable occurs in an expression, the corresponding value should be loaded from memory. \\ \hline 
		\includegraphics[align=t,width=250px]{cswk/3-if.png} &
		If the condition of an if statement is true (\emph{i.e.} any non-zero value), then the body of the if clause should be executed. \\ \hline
		\includegraphics[align=t,width=250px]{cswk/4-else.png} &
		If the condition of an if statement is false (\emph{i.e.} it evaluates to zero), then the body of the else clause should be executed. \\ \hline 
		\includegraphics[align=t,width=250px]{cswk/5-ifelse.png} &
		If there are if else clauses present, their conditions should be checked in order after that of the main if clause. If one of them is true, then the corresponding body should be executed. \\ \hline
		\includegraphics[align=t,width=250px]{cswk/6-repeat.png} &
		Repeat statements evaluate an expression to determine how many times they should run. The body of the repeat statement is then executed that many times. \\ \hline 
		\includegraphics[align=t,width=250px]{cswk/7-divbyzero.png} &
		If a division by zero is attempted, the corresponding error should be returned. \emph{I.e.} \haskellIn{Left DivByZeroError} \\ \hline
	\end{longtable}
\end{center}
There are some details to look out for:
\begin{itemize}
	\item Internally, logic operators should evaluate to $0$ if false or a non-zero value if true. All numeric values other than $0$ should be treated as true.
	\item If an attempt is made to read from a variable which is not in the memory, then the corresponding error should be returned.
	\item Expressions can be nested arbitrarily deep and expressions of arbitrary complexity may appear in any place where expressions are expected.
	\item Errors can arise almost anywhere and should be propagated properly.
\end{itemize}

Running \bashIn{stack test} will give you a rough indication of how complete your solution is. Running \bashIn{stack bench} will benchmark your code.

\section{Marking \& submission}

This coursework is worth 25\% of the overall module mark. It will be marked out of 100\% as follows:
\begin{itemize}
	\item 20\% for \emph{correctness}. You gain full marks here if all parts of the coursework have been attempted and are correct. You may use \bashIn{stack test} as a rough indication for whether this is the case, but there are some things the unit tests do not test for, so you should construct programs in the scratch clone and ensure that everything works as described.
	\item 20\% for \emph{documented understanding}. You should document your code with comments and explain how it works. You gain full marks if all code is documented and explained sufficiently well so that someone who is unfamiliar with your code can understand it.
	\item 20\% for \emph{elegance}. Definitions should be concise and readable, new functions should be introduced where needed, existing library functions used when applicable, monads used where possible, etc. 
	\item 20\% for \emph{performance and efficiency}. You will do well here if you use sensible data structures and your functions perform as little redundant computation as possible. You can test performance by running \bashIn{stack bench} on different versions of your code to see how they compare. 
	\item 20\% for \emph{improvements and extensions}. This is an opportunity for you to demonstrate creativity and advanced understanding. You could achieve this in many different ways, such as adding additional unit tests, functionality, improved algorithms, etc. You may wish to modify \texttt{\small exe/Main.hs} as well as other source files or even add new ones. You could also prove some properties about your interpreter on paper. The amount of marks awarded will depend on the complexity and creativity of your extension(s) and improvement(s).
\end{itemize}
Submit a \texttt{\small .zip} or \texttt{\small .tar.gz} archive of the whole, completed project (not just \texttt{\small Interpreter.hs}) through Tabula by \deadlineTwoTime\ on \deadlineTwoDate:

\begin{center} 
\url{\submissionTwoURL}
\end{center}

\cleardoublepage
\chapter{Solutions}

This section contains model answers for selected exercises along with descriptions of how they could be derived. You should only read these for revision purposes or once you have completed the exercises yourself. Remember that it is better for you to ask for help in figuring something out than to just jump to the solutions.

\input{labs/sol3.tex}
\input{labs/sol5.tex}
\input{labs/sol6.tex}

\end{document}
